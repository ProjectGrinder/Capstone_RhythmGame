\refstepcounter{section}
\section*{Appendix \thesection \, \textbar \vspace{0.5em} Feasibility Analysis}
\label{sec:appendix-feasibility}
\addcontentsline{toc}{section}{Appendix \thesection. Feasibility Analysis}%

\subsection*{1. Technical Feasibility}
Each member of our team has skills and knowledge that are required in different areas of the project.
This includes C++ programming, system architecture design, game design, graphics programming, and team management.
This team structure ensures that all aspects of the project are covered by knowledgeable individuals.
The project utilizes C++20, DirectX 11, and Windows API, which are well-documented and widely used technologies, with no other third-party libraries required.
Using official libraries ensures compatibility and stability for our target platform, Windows.
Additionally, using GoogleTest for unit testing allows us to maintain system reliability and detect unexpected bugs throughout the development process.

\subsection*{2. Schedule Feasibility}
Our development process will follow an agile development methodology.
The project is broken down into manageable tasks assigned to team members based on their expertise.
Our team also plans to hold weekly meetings to discuss progress, address problems, and adjust plans as necessary.
In addition, following the agile methodology, each member is not locked to a single duty and can take multiple responsibilities depending on the current needs of the project.
This flexibility ensures that the project can adapt to changing requirements and challenges with no significant delays.
Upon consideration of estimated workloads, the timeline appears achievable and is expected to be met within the second semester.

\subsection*{3. Organizational Feasibility}
The project aims to provide tools and resources to help users create interactive media efficiently.
This aligns with the goals of our academic institution, which emphasizes practical and beneficial applications of technology in the industry.
Moreover, ECS architecture is not only limited to game development but can be adapted in other fields of interactive media.
Here are some examples:
\begin{itemize}
    \item Animation\\
    ECS can easily manage media elements such as characters, props, and effects, allowing a convenient process of animating and modifying elements.
    \item Dynamic UI\\
    ECS can support dynamic user interface in applications with various buttons, sliders, panels, and other elements that can have multiple properties.
    \item Simulation\\
    Due to its performance and modularity, ECS is beneficial to accurate and realistic behaviors in media simulations.
    \item Real-Time Data Visualization\\
    Apart from managing data visuals, ECS can also be used for handling live data with constant updates.
\end{itemize}