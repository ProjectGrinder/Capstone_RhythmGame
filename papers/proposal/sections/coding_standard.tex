\refstepcounter{section}
\section*{Appendix \thesection \, \textbar \vspace{0.5em} Coding Standard}
\label{sec:appendix-coding-standard}
\addcontentsline{toc}{section}{Appendix \thesection. Coding Standard}

All source code in this project follows a consistent formatting style defined by a custom Clang-Format configuration.
The configuration is based on the LLVM coding style, with modifications for improved readability, clearer brace structure, and consistent indentation across the codebase.
\subsubsection*{1. Base Style}
\begin{itemize}
\item
The formatting standard is derived from the LLVM style guide, chosen for its balance between compactness and readability.
\item
The configuration file is provided as .clang-format in the project root directory and automatically enforced via build or commit hooks.
\end{itemize}
\subsubsection*{2. General Formatting}
\begin{itemize}
    \item
    Indentation: 4 spaces per level; tab width is also 4 spaces. Tabs are never used for indentation.
    \item
    Column limit: 120 characters per line to preserve readability on modern displays.
    \item
    Namespace indentation: All namespaces are indented to clearly scope internal structures.
    \item
    Access modifiers: public, protected, and private are indented for better visual separation from member declarations.
\end{itemize}
\subsubsection*{3. Brace and Block Rules}
\begin{itemize}
    \item
    Braces follow a custom “always break” style, placing braces on a new line for classes, structs, enums, functions, and control statements.
    \item
    Empty braces (e.g., {}) are never split across lines.
    \item
    if, else, catch, while, and lambda bodies always begin on a new line to emphasize logical structure and reduce ambiguity.
\end{itemize}
\subsubsection*{4. Alignment and Spacing}
\begin{itemize}
    \item
    Alignments: Operands, braces, and trailing comments are aligned to maintain vertical clarity.
    \item
    Assignments and declarations are not force-aligned to avoid inconsistent whitespace when editing.
    \item
    Spacing:
    \begin{itemize}
    \item
    No extra spaces inside parentheses or template angle brackets.
    \item
    One space after C-style casts (e.g., (int) x).
    \item
    No spaces before range-based for loop colons or conditional parentheses.
\end{itemize}
\end{itemize}
\subsubsection*{5. Line Breaking}
\begin{itemize}
    \item
    Function and template declarations break long parameter lists and template definitions across lines for readability.
    \item
    Constructor initializers are aligned vertically and placed one per line when multiple exist.
    \item
    Arguments and parameters are never bin-packed — each remains on its own line when wrapping occurs.
\end{itemize}
\subsubsection*{6. Includes and Imports}
\begin{itemize}
    \item
    Include order is automatically sorted into three priority groups:
    \begin{enumerate}
    \item
    System headers (\textless\ldots\textgreater)
    \item
    Local headers (\textquotedblleft\ldots\textquotedblright)
    \item
    Other includes (project-specific or third-party)
    \end{enumerate}
    \item
    Include sorting ensures deterministic builds and improves merge conflict resolution.
\end{itemize}
\subsubsection*{7. Empty Lines and File End}
\begin{itemize}
    \item
    No more than two consecutive empty lines are kept.
    \item
    A newline is automatically inserted at the end of each file.
    \item
    Preprocessor directives are indented before the hash symbol to match surrounding indentation levels.
    \end{itemize}
    \subsubsection*{8. Code Style Philosophy}
    This formatting standard prioritizes:
    \begin{itemize}
    \item
    Readability — through consistent brace placement and indentation.
    \item
    Maintainability — by avoiding complex line packing and ensuring clear code structure.
    \item
    Cross-platform consistency — the same formatting rules apply to all C and C++ source files regardless of platform or compiler.
\end{itemize}