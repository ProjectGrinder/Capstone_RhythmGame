\section{Introduction}
\label{sec:introduction}

\subsection{Background \& Motivation}
\label{subsec:background-and-motivation}

In recent years, interactive media has become increasingly more integrated into modern society.
It has appeared within games, movies, applications, and Virtual Reality (VR) with usage covering
many fields, including within academic and medical contexts.
As such media continues to develop in scale and complexity, its performance demands have intensified.
The growth in hardware performance is not enough to keep up with this demand, creating a challenge for
creators and system designers alike.
\\\\
Current interactive media is usually created by relying on game engines, such as Unity, Godot, and Unreal Engine,
as a software framework.
However, these game engines were originally designed for game development and were not built for the more
general applications.
This mismatch motivates the need to create such a system that is built optimally for these new demands,
while remaining highly optimized in terms of performance for various tasks.

\subsection{Problem Statement}
\label{subsec:problem-statement}

Existing game engines are not suitable for the creation of general-purpose interactive media.
Their architecture uses the traditional object-oriented programming (OOP) paradigm, which introduces
certain constraints and limitations on flexibility, scalability, and performance outside game contexts.
\\\\
In response, many companies have resorted to developing proprietary systems from scratch to meet their specific requirements.
This practice, however, leads to delays in development while accruing unnecessary costs.
The problem, therefore, is that there lacks a sufficiently versatile and high-performance framework that can
efficiently support the development of general-purpose interactive media.

\subsection{Objectives}
\label{subsec:objectives}

\begin{enumerate}
    \item To analyze the limitations of existing game engine architectures, particularly in relation to general-purpose interactive applications.
    \item To explore alternative architectural paradigms that could offer improved performance and flexibility over conventional OOP-based designs.
    \item To propose and develop a framework that can streamline the creation of interactive media while reducing development costs and maintaining high performance across various domains.
\end{enumerate}

\subsection{Scope \& Limitations}
\label{subsec:scope-and-limitation}
This project intended to create high precision software.
In order to achieve this goal, we need to have control on our system as much as possible.
That's mean everything on this system will be relied on only Operating System's library and
C++ standard library (STL) function that is proved to be zero cost abstraction according to
International Organization for Standardization (ISO) but if specify to be up to implementation
we will refer to Microsoft Visual C++ implementation of STL\@.
\\\\
Scope of this paper will be implementation of rhythm game and bullet hell and
compete on Talent Showcase competition by Thai Game Software Industry Association (TGA)
using fewer libraries as much as possible.

\subsection{Expected Benefits}
\label{subsec:expected-benefits}

\begin{enumerate}
    \item A clearer understanding of current game design architectures and its limitations on the development of general-purpose interactive media.
    \item A structured analysis of alternative design paradigms that may overcome these limitations.
    \item A practical framework that simplifies the development of interactive systems across multiple application domains.
    \item Enhanced adaptability and performance in future interactive media platforms.
\end{enumerate}