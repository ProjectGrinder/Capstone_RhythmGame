\section{Introduction}
\label{sec:introduction}
In recent years, interactive media is found in various places throughout digital society,
such as within games, visual effects in movies, applications, or Visual Reality (VR)
with usage covering many fields.
For example, Virtual Reality as a recovery tool in a medical system.
Because of this demand, the performance requirements are becoming more prominent.
But on the hardware side, the growth of performance
is far less than the demand.
\\\\
The current trend of creating interactive media is using game engines such as Unity, Godot, or
Unreal to create this type of media.
But these systems weren't built with this general usage in mind, but rather were specialized for creating games.
Because of this, these engines might not be suitable for widespread everyday usage with such high demands.
Many companies decided to write their own systems from the ground up, which led to slower development while accruing
unnecessary costs.
\\\\
The core ideas of these game engine systems rely on Object Oriented Programming (OOP) which provide easier to understand
software architecture.
But with the implementation of OOP in many programming language, usually tends to create a system where every calls are
made with pointer calls or indirection calls which create an issue on compiler understanding the intention of code
clearly which leads to compiler unable to aggressively optimize the code.
\\\\
This gave rise to another programming paradigm where we just have raw data and a set of function to manipulate it.
Entity Component System (ECS) which is another style of writing software in easy to understand way.
But the problem is that most of ECS implementation rely on reflection which can also hurt the performance greatly.
\\\\
But with C++20 which provide a self write code on compile time (templating), this can be achieved on compile time
which provide compiler total understanding of software code and it's intention which will give compiler the ability
to aggressively optimize the code.
\\\\
Also in current market, most of the hardware are now multithread support which means we can write a code in such a
way where we can send the subsystem to multiple threads in order to reduce delay.
This will ensure the fully utilization of CPU\@.
\\\\
On GPU side, we can write software in such a way where the data itself got turn from Array of Struct (AoS) to Struct of Array (SoA).
And using the idea of GPU offloading in order to optimize a data transformation and continue to render the data to the screen.
\\\\
These situations raise the question \textemdash whether it is possible to create a system that can be used to generate such media
and highly optimal for various tasks while also remaining highly optimized.
\\\\
To evaluate the precision of such software, a rhythm game is a great and representative subject
due to its strict needs for high-precision inputs ensuring good responsiveness and a better impression for the user.
Additionally, a bullet hell game, a video game genre revolved around having a vast amount of projectiles on the screen at the same time,
serves as an excellent demonstration of performance under stress for the software.
\\\\
To evaluate the generality of such software, both of the genre, being highly different in logics, are being integrated together seamlessly
in order to showcase the versatility of the software alongside with 2D side scrolling gameplay.
\\\\
Lastly, to validate if the system is viable from a business standpoint, we will develop this game
to compete in the Game Talent Showcase by Thailand Game Association (TGA).

\subsection{Background}
\label{subsec:background}
% Background goes here

\subsubsection*{Entity-Component-System Architecture}

Entity-component-system (ECS) is a software architectural pattern mostly used in game development.
Its mechanisms are characterized by the entities, composed of components of data,
and the systems which operate upon those components.
The behaviors of these entities are modified at runtime by systems modifying, adding, or removing the components
attached to said entities.
\\\\
Implementations of ECS already exist, even in production contexts.
One of the more popular implementations is EnTT~\cite{ValtoLibraries_EnTT}, a header-only C++ library for game programming.
This library has been used to develop many game titles, the most major of which is Minecraft.
Another popular implementation is Bevy ECS~\cite{Bevy_Engine}, a data-driven game engine built in Rust.
\\\\
However, most existing ECS libraries use runtime component registration, perform type-erased iteration, or resolve
system dependencies at runtime, or some combination of the above.
While they often optimize queries via template views or code specialization, they cannot fully eliminate
runtime overhead associated with registration, lookup, or caching.
Experimental systems like Vittorio's ECST~\cite{vittorio} achieve deep compile-time specialization
but are not production-ready due to ambiguity in naming, sparse documentation, and hidden dependencies.
\\\\
We want to combine the best of both worlds \textemdash to create an ECS framework that registers and determines component types
and system dependencies at compile-time, while ensuring code flexibility and production-ready performance, complete
with rigorous documentation.

\subsubsection*{Market Research}

The computer software market has been growing for many years, and software performance continues to develop 
as the demand for advanced software systems increases in various industries.
We have looked into a report of the High Performance Computing Software Market~\cite{HPC_Software_Market_Research} to
understand the possible trend of software engineering in the future.
According to the research, the market is forecasted to grow from 45.5 billion USD in 2024 to 99 billion USD by 2035.
\\\\
High performance computing (HPC) software has become essential for optimizing complex processes to ensure high-speed operations.
With the increased use of advanced technology, including artificial intelligence, machine learning, simulations, and many more 
modules and services, HPC software is necessary for these technologies to function properly.
Additionally, organizations and companies are also interested in software efficiency to lower the costs of computing resources, 
and implementation with cloud services.
This can be marked as another direction of the market which focuses on sustainable and energy-efficient products.
Therefore, an ideal software system for the industry should be able to provide high performance while simultaneously being 
optimized and efficient to reduce operation costs and resources.
\\\\
Upon considering regional valuation, North America is the leader of the global HPC market, valued at 20.0 billion USD in 2024, 
with the U.S.A.being the key driver of the market due to strong demand for highly advanced software and contributions from
various tech companies, including Microsoft, IBM, AWS, and NVIDIA\@.
Following North America is Europe, with a valuation of 11.0 billion USD in 2024, and Asia-Pacific region (APAC), valued at 
10.0 billion USD in 2024.
The global market trend shows that many countries around the world are investing in HPC software and companies are innovating 
new technologies to capitalize on the growing demands.
\\\\
Our project does not cover only the development of high-performance software, but also its utilization.
Therefore, we decided to research a performance-reliant medium for computer software: video games.
We have looked into a report of Gaming Software Market Analysis 2025--2029~\cite{Gaming_Software_Market_Analysis} to understand the direction of
the gaming industry in different regions.
The market has been growing significantly due to advancements in game engines which support mobile and tablet games, along with
increasing popularity in eSports, short for electronic sports, which is a form of competition using video games, creating new
revenue streams for the market.
Game engines are being optimized to improve performance and graphical processes.
To capitalize on market opportunities, gaming software companies must keep up in research and development of technologies to
provide new experiences to consumers.
\\\\
According to Technavio's analysts, the Asia Pacific (APAC) region is estimated to contribute 43\% to the growth of the global
market during the forecast period.
With China, Japan, and India as leading countries, APAC dominates the market due to the largest number of mobile gamers,
in addition to the increase in smartphone usage and high-speed internet services.
The advancements in game engines serve as the primary driver for the rise of the gaming software market, driven by the
evolution of technologies in rendering, animation, graphics, augmented/virtual reality (AR/VR), and artificial intelligence (AI).
Game development tools, including engines, continue to be a crucial investment for businesses in the industry.

\subsubsection*{Market Research}



\subsection{Problem Statement}
\label{subsec:problem-statement}

\subsection{Objectives}
\label{subsec:objectives}

\subsection{Scope \& Limitations}
\label{subsec:scope-and-limitation}

\subsection{Expected Benefits}
\label{subsec:expected-benefits}