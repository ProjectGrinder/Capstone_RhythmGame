\subsection{Collaborative Plan}
\label{subsec:collaborative-plan}
The way we plan to work together is to have a group meeting for updating progress and planning on Friday of every week.
We also have a meeting with the advisor on Thursday of every week.
If there is a need to be absent, every person needs to clearly state so beforehand.
Each system planning will be planned as sprints with durations of two weeks per each iteration.
After every iteration, we will have a reflection on the first Friday after the end of iteration.
For meeting, everyone on the team has selected to have meetings in online format on Discord.
\\\\
For project source code management, we will be using Git Version Control System and using GitHub as
a remote repository.
The project source code will contain only system source code, game source code, document, report and report image only.
For game resources will need to be downloaded outside the repository.
\\\\
We will be using Cmake as a build system to allow team members to work easily with any text editor.
We will also be using clangd as a formatter in order to have a consistent coding style.
MSVC is used as a main system compiler such that we target Windows Operating System specifically.
Although we allow any text editor to use within this project, Clion by JetBrains is preferred.

\subsection{Task Coverage}
\label{subsec:task-coverage}

\subsection*{Project Lead}
This is a person who will manage and plan the project.
The main task of this person is to create a systematic plan for the project which is tangible and able to lead to finish product.
Not only that, this person also needs to communicate and also contact insiders and outsiders as needed.
Also, this person must be able to understand every person on a team and be able to find a suitable solution to make
the work go smoothly as much as possible.
\\\\
Since this team is a small team, the project lead must be able to direct the project direction and understand the image
of the final product so that they can achieve the most beneficial outcome for the team.

\subsection*{System Software Architecture}
This is a person who is responsible for our system design.
The responsibility for this person is that they must be able to design an architecture which is
suitable for the needs of the software.
They must also create a good work environment for the team to work
with using this system in the future.

\subsection*{Gameplay Design}
This person is the person whose responsibility is to design a game which can test whether
the system is working as intended or not.

\subsection*{Game Software Engineer}
This is a person whose task is to implement the game according to gameplay design within the
structure planned by Gameplay Software Architecture.

\subsection*{Graphic Software Engineer}
This person is a person who works on creating graphic which has high fidelity and also works
on optimization the graphic section of the software such as GPU Memory Management.

\subsection*{Operating System Software Engineer}
This is a person who creates an Application Programming Interface (API) to communicate with
the operating system safely and optimally.
Also, this person must ensure the software security and memory safety.

\subsection*{User Level Software Engineer}
This is who creates an API for users to write software on this system.
The person in charged must be able to create a way that is easier to implement while
also easy to use.

\subsection*{Gameplay Software Architecture}
This person is a person who designs the system flow of a game to ensure good user experience
and also optimal data flow within the game system.

\subsection{Workload Balance}
\label{subsec:workload-balance}
    Since this project is really large and requires highly specialized personal, each task was assigned
    to each person with intent for them to be the master of that aspect of the system.
    Each of the tasks has their own set of challenges and requires very deep understanding.
    This means every person got to have their own hard and easy tasks where on higher level management can be inferred
    as a task is already balanced because of the need of specialty.